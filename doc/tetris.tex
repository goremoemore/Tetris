\documentclass{article}
\usepackage[utf8]{inputenc}
\usepackage[T2A]{fontenc}
\usepackage[russian]{babel}

\title{BrickGame v1.0 aka Tetris}
\author{Автор: Lynnesys}

\begin{document}
\maketitle

\section*{Краткое описание}
    Представлена реализация игры "Тетрис" на языке программирования С.

В рамках проекта выполнены все минимально необходимые функции игры:

    - автоматическое движение фигур вниз (скорость изменяется от уровня);

    - движение фигур влево/вправо;

    - поворот фигур;

    - моментальное опускание фигуры;

    - очистка заполненных линий и подсчет очков на основании их количества;

    - сохранение наилучшего результата и имени игрока в базе данных.


\section*{Старт игры и управление}
Перед началом игры необходимо ввести имя игрока.

Если такое имя уже существует в базе, будет взят наилучший результат игрока, в противном случае - создается новый пользователь с результатом "0".

Для перемещения фигуры влево используется стрелка "влево".
Вправо - стрелка "вправо".

Для моментального опускания фигуры вниз используется стрелка "вниз".

Для поворота фигуры используется "пробел".

После завершения игры можно посмотреть список лучших 10-ти игроков, нажав клавишу s.

Чтобы приостановить игру (пауза) нужно нажать английскую букву р.

Для выхода из игры - ESC.


\section*{Подсчет очков и система уровней}
На основании количества одновременно очищенных линий игроку начисляются очки:

- 1 линия -- 100 очков;

- 2 линии -- 300 очков;

- 3 линии -- 700 очков;

- 4 линии -- 1500 очков.


Каждые 600 набранных очков повышают уровень игры - фигуры падают быстрее. Всего в игре 10 уровней.


\section*{Компиляция}
Для запуска игры необходимо, находясь в директории src, в терминале вызвать команду "make run".


Данное действие автоматически очистит (при наличии существующих файлов), перекомпилирует и запустит игру.


Для корректной работы сохранения очков должна быть установлена СУБД SQLite3.

\end{document}